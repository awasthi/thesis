\part{Conclusion and Perspectives}
\label{end con}
\chapter{Conclusion}
In this thesis, two parts of works have been dedicated. In the first part, we present some scheme for a special class of PRNGs based on chaotic iterations (CIPRNG). They are proven to be cryptographically secure, and evaluated statistically. FPGA is also applied to enhance the CIPRNG performance. Then some applications taken from the field of cryptography are finally proposed. In the second part, a preparation for the application of CI into broadband optoelectronic entropy sources to generate ultra fast random bits has been done, using the MSB (or equivalently 1-bit ADC conversion) has been proven the way to keep a deterministic origin in the generation of a random bit sequence. 

\section{CI method}
Chaos, being a special class of nonlinear dynamics, has aroused a lot of interests since its emergence and formulation in science these last decades. Various distinct  behaviors, such as random-like exhibition, sensitive dependence to initial conditions, and unpredictability, together with their inherent determinism and simplicity of realization, justify this interest. More recently, some researchers have explored the possibility to apply these chaotic dynamics for cryptographic applications and designs. To become an element to take into account in cryptography, the nature of randomness and chaos have been connected. Thus interrelations between chaos-based random number generators and cryptographic systems have been investigated. However, such investigations have led to controversy, as people involved in cryptography not always understand the tools and approaches of the nonlinear community, and these latter do not deal rigorously (mathematically) with security, often forgetting to take into account one century of formalization and advances in cryptography.

These difficulties are reinforced by the fact that tools manipulated in nonlinear science are often inadequate, far from the targeted applications in computer science. Indeed, in most of the designs, the generation of chaos is obtained by a recurrence relationship or a set of state differential equations, in which perfect model is assumed. It always requires a continuous space domain so that the actual chaotic dynamics can be observed. However, this space domain implies the use of real-number or infinite bit representation in system realization. As pinpointed by some researchers, the digital dynamical properties of a chaotic system will be far different from its continuous ones. For instance, when some piecewise linear maps are realized in finite computing precision, some severe problems such as short cycle length, non-ideal distribution, and high correlation, have been observed and reported. Having these issues in mind, it seems more realistic and practical to consider a chaotic system realized in finite 
precision domain. In previous work, a a family of new chaotic pseudo-random bit generators called CI PRNGs is proposed, which is going to try to fill the gap.

In this part of thesis, the CI PRNGs method is developed and evaluated. These generators are based on discrete chaotic iterations, which satisfy the well respected Devaney's definition of chaos. The rigorous framework for generate chaotic random sequence in \cite{bibtexwangqianxue} has been developed and renewed. Firstly the researches of the CIPRNG Version 1 method has been deepen; Then and new version: CIPRNG Version 3 is represented by using lookup table method; At last, CIPRNG Version 4 which is specifically designed for FPGA hardware is indicated. All of these version have been proven to be cryptographically  secure in some cases.

The randomness and disorder generated by these versions CIPRNG algorithms have been widely evaluated, this depends on both the proof of theoretical properties and the scores on numerous statistical tests.  Various statistical tests are available in the literature to check empirically the statistical quality of a given sequence. The most famous and important batteries of tests for evaluating PRNGs are, namely: the TestU01, NIST, and DieHARD batteries. To evaluate the propositions of this thesis, a comparative study between various generators and our own PRNG has been carried out and statistical results have been highlighted.  They all lead to the conclusion that these generators can be considered as candidates for a large variety of applications in computer science, and in the security field too. By using a secure PRNG as input, the CIPRNG methods have been proven that the cryptographically secure properties could be succeed. The CI algorithm has been also modified to use in FPGA, which leads a huge speed improvement.

The PRNG we have proposed is based on the so-called discrete chaotic iterations. It is a composite generator that combines the features of two other PRNGs.  The intention of this combination is to accumulate the effects of chaos (brought by the chaotic iterations) and randomness (taken from the inputted generators). The results of comparative test parameters confirm that the proposed CI PRNGs are all able to pass these tests, and improve the statistical properties relative to each generator taken alone. Also ten famous classic PRNGs are applied to show the CI method is able to enhance the statistical performance, and this provide a technology to make PRNG withstand the attacks in future.

This is why application examples in cryptography are finally given at the end of this part. We have more specifically study the possible use of such a family of generators for digital watermarking. Security has not been investigated, only robustness to some frequency and geometric attacks has been evaluated. This case study enables us to precise the details of the algorithm and to give a concrete illustrative example of the interest to possess a  generator being both random and chaotic. The proposed CI PRNGs can be used to fulfill needs of image encryption. Detailed analyses show that they can provide high level of security.

\section{Optical source}
Recently, for unprecedented high bit rate of random sequences , very attractive solution has been given by photonic broadband signal to generate random numbers. There are two different origins to extract bit streams from photonic analogue waveforms according to nowadays physical systems: the source of randomness explicitly stems from noise \cite{li:OL11,wetzel:ox12} and deterministic chaos \cite{fast,ultrafast2009}. In this context, we have proposed in the thesis to analysis a related method, though an analysis of the deterministic origin of the chaos-based photonic RNGs proposed in \cite{ultrafast2009, ultrafast2010}. The particularity of this method, is that it mixes both photonic chaos, but also a significant digital post-processing before the extraction of a final bit stream with excellent randomness. Here in the thesis, this method on other photonic sources of analogue entropy has been redone. To be noticed, in one hand, the random streams is extracted by a mainly deterministic electro-optic phase chaos 
generator, which recently is used to demonstrate the currently fastest analogue chaos communication at 10~Gb/s. In the other hand, the scheme with another photonic source of entropy is also considered:  the photonic source of entropy can be trustfully considered as of mainly noisy origin, without any deterministic and controllable compound: its origin is typically attributed to intrinsic diode laser RIN (relative intensity noise), photodiode detection noise, and electronic amplifiers noise (check in Fig.~\ref{opto_RNG}). During the processing, we found that the method proposed in \cite{ultrafast2009,ultrafast2010} gave fully comparable results in terms randomness quality of the generated bit stream. This is fully consistent with the questioning already addressed in \cite{williams:OE10,hirano:OE10} about the actual origin, noise or chaos, of the random bit stream provided by this method. More than that, we have proposed analysis of this method on the basis of standard signal theory and sampling theory. The 
conclusion of our analysis strongly support that
the method is essentially exploiting the noisy compound always present in a photonic signal, would it be with (chaos) or without (noise) deterministic motion. Our analysis has highlighted in the post-processing bit extraction method, two dominating mechanisms leading to a high quality random bit stream. One is related to the natural spectral mixing occurring when aliasing is involved, another post-processing which we have called DSD for distant sample difference, was also shown to contribute to the randomness quality of the final bit stream. At last, attached together with LSBs retaining only, it can shown that the the small amplitudes which are mainly dominated by noise source have given an important role to the outputs. To develop an evidence of the actually most important role of the noise instead of the deterministic chaotic motion, analyzing the binary entropy rate for each selected bit from LSB to MSB, without or with one or several DSD post-processing are proposed. It is only found the existence of a 
non zero memory time in parts closed to MSB, for a small number of DSD post-processing. This configuration which reveal a signature of the deterministic origin via the evolution of the binary entropy, is the opposite one with respect to the choice proposed in \cite{ultrafast2009, ultrafast2010}. It thus confirms that negligible deterministic origin (and thus the chaotic light motion) actually enters in the randomness of the final bit stream.

Applying the MSB (or equivalently 1-bit ADC conversion) is according to us the solution that processing a deterministic origin random bit sequence generation. This has provided the preparation for applying CI in photonic signal to generate deterministic random streams. At last two examples of mixing CI and MSB of optoelectronic chaotic signal is given, the results of NIST test suite show the improvements of randomness, in future, the practical implementation is looking forwarded. 


\chapter{Perspectives}
In this Section, we mention a few question which we would like to examine further.
\section{CI method}
We considered in detail the CIPRNG model using a random update function . It
would be interesting to know if we can use the same approach for other more primitive input sequences and keep their properties. Cryptographically secure PRNGs are an example of primitives which are currently in the center of attention and make the output CIPRNG be a Cryptographically secure PRNG. And, we will continue to try to improve the speed
and security of this PRNG, by exploring new strategies and
iteration functions. Its chaotic behavior will be deepened by
using the numerous tools provided by the mathematical theory
of chaos. A larger variety of tests will be considered to compare this
PRNG to existing ones, and a cryptanalysis of our generator will be
proposed. The chaotic behavior of the proposed generator will be deepened
by using the various tools provided by the mathematical theory of chaos. Additionally a probabilistic study of
its security will be done. Lastly, new applications in computer
science will be proposed, especially in the Internet security
field.

\section{Optical source}
According to
us, the way to keep a deterministic origin in the generation of a
random bit sequence is using the MSB (or equivalently 1-bit ADC conversion). In future work, we will concentrate on this approach for  the
bit extraction method from the original photonic chaotic
waveform. Finally, the most
difficult task will be to design a proper coupling scheme with a
binary bit stream, so that distant random sequences can be
synchronized and used for cryptography.