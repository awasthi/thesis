\chapter{Appendix}
\label{appendix}
In a cryptographic context, a pseudorandom generator is a deterministic algorithm $G$ transforming strings into strings and such that, for any seed 
$s$ of length m, $G(s)$ (the output of $G$ on the input $s$) has size $l_G(m)$ with $l_G(m) > m$. The notion of secure 
PRNGs can now be defined as follows.

\begin{definition}
\label{CSPRNG}
A cryptographic PRNG $G$ is secure if for any probabilistic polynomial time algorithm D, for any positive polynomial p, 
and for all sufficiently large m's,  
\begin{equation}
\left|Pr[D(G(U_m))=1]-Pr[D(U_{l_G(m)})=1]\right|<\frac{1}{p(m)},
\end{equation}
where $U_r$ is the uniform distribution over $\{0, 1\}^r$ and the probabilities are taken over $U_m$, 
$U_{l_G(m)}$ as well as over the internal coin tosses of $D$.
\end{definition}

Intuitively, it means that there is no polynomial time algorithm that can distinguish a perfect uniform 
random generator from $G$ with a non negligible probability. Note that it is quite easily possible to change 
the function $l$ into any polynomial function $l'$ satisfying $l'(m)>m$. 

Let us consider now the XOR CIPRNG defined by:
\begin{equation}
\left\{
\begin{array}{l}
x^0 \in \llbracket 0, 2^\mathsf{N}-1 \rrbracket, S \in \llbracket 0, 2^\mathsf{N}-1 \rrbracket^\mathds{N} \\
\forall n \in \mathds{N}^*, x^n = x^{n-1} \oplus S^n,
\end{array}
\right.
\label{equation Oplus}
\end{equation}
This PRNG being very closed to classics block cipher modes of operation, it is probable that, if
$S$ satisfies Definition~\ref{CSPRNG}, then it should be the case too for the resulting XOR CIPRNG.

A proof of this statement could be written using a contraposition approach: 
if we suppose that a distinguisher (polynomial time probabilistic algorithm) $D$ exists for the XOR CIPRNG, then we
can use it to construct another distinguisher $D'$ for $S$. Thus, if no 
distinguisher exists for $S$, the XOR CIPRNG will be cryptographically secured.

To have a complete manuscript, we give in what follows a demonstration of this 
fact written by colleagues of
my team (with their agreement). 

\begin{proof}
The generation schema developed in (\ref{equation Oplus}) is based on a
pseudorandom generator. Let $H$ be a cryptographic PRNG. We may assume,
without loss of generality, that for any string $S_0$ of size $N$, the size
of $H(S_0)$ is $kN$, with $k>2$. It means that $\ell_H(N)=kN$. 
Let $S_1,\ldots,S_k$ be the 
strings of length $N$ such that $H(S_0)=S_1 \ldots S_k$ ($H(S_0)$ is the concatenation of
the $S_i$'s). The cryptographic PRNG $X$ defined in (\ref{equation Oplus})
is the algorithm mapping any string of length $2N$ $x_0S_0$ into the string
$(x_0\oplus S_0 \oplus S_1)(x_0\oplus S_0 \oplus S_1\oplus S_2)\ldots
(x_o\bigoplus_{i=0}^{i=k}S_i)$. One in particular has $\ell_{X}(2N)=kN=\ell_H(N)$. 

We claim now that:
If $H$ is a secure cryptographic PRNG, then $X$ is a secure cryptographic
PRNG too.


The proposition is proven by contraposition. Assume that $X$ is not
secure. By Definition, there exists a polynomial time probabilistic
algorithm $D$, a positive polynomial $p$, such that for all $k_0$ there exists
$N\geq \frac{k_0}{2}$ satisfying 
$$| \mathrm{Pr}[D(X(U_{2N}))=1]-\mathrm{Pr}[D(U_{kN}=1]|\geq \frac{1}{p(2N)}.$$
We describe a new probabilistic algorithm $D^\prime$ on an input $w$ of size
$kN$:
\begin{enumerate}
\item Decompose $w$ into $w=w_1\ldots w_{k}$, where each $w_i$ has size $N$.
\item Pick a string $y$ of size $N$ uniformly at random.
\item Compute $z=(y\oplus w_1)(y\oplus w_1\oplus w_2)\ldots (y
  \bigoplus_{i=1}^{i=k} w_i).$
\item Return $D(z)$.
\end{enumerate}


Consider  for each $y\in \mathbb{B}^{kN}$ the function $\varphi_{y}$
from $\mathbb{B}^{kN}$ into $\mathbb{B}^{kN}$ mapping $w=w_1\ldots w_k$
(each $w_i$ has length $N$) to 
$(y\oplus w_1)(y\oplus w_1\oplus w_2)\ldots (y
  \bigoplus_{i=1}^{i=k_1} w_i).$ By construction, one has for every $w$,
\begin{equation}\label{PCH-1}
D^\prime(w)=D(\varphi_y(w)),
\end{equation}
where $y$ is randomly generated. 
Moreover, for each $y$, $\varphi_{y}$ is injective: if 
$(y\oplus w_1)(y\oplus w_1\oplus w_2)\ldots (y\bigoplus_{i=1}^{i=k_1}
w_i)=(y\oplus w_1^\prime)(y\oplus w_1^\prime\oplus w_2^\prime)\ldots
(y\bigoplus_{i=1}^{i=k} w_i^\prime)$, then for every $1\leq j\leq k$,
$y\bigoplus_{i=1}^{i=j} w_i^\prime=y\bigoplus_{i=1}^{i=j} w_i$. It follows,
by a direct induction, that $w_i=w_i^\prime$. Furthermore, since $\mathbb{B}^{kN}$
is finite, each $\varphi_y$ is bijective. Therefore, and using (\ref{PCH-1}),
one has
$\mathrm{Pr}[D^\prime(U_{kN})=1]=\mathrm{Pr}[D(\varphi_y(U_{kN}))=1]$ and,
therefore, 
\begin{equation}\label{PCH-2}
\mathrm{Pr}[D^\prime(U_{kN})=1]=\mathrm{Pr}[D(U_{kN})=1].
\end{equation}

Now, using (\ref{PCH-1}) again, one has  for every $x$,
\begin{equation}\label{PCH-3}
D^\prime(H(x))=D(\varphi_y(H(x))),
\end{equation}
where $y$ is randomly generated. By construction, $\varphi_y(H(x))=X(yx)$,
thus
\begin{equation}%\label{PCH-3}      %%RAPH : j'ai viré ce label qui existe déjà, il est 3 ligne avant
D^\prime(H(x))=D(X(yx)),
\end{equation}
where $y$ is randomly generated. 
It follows that 

\begin{equation}\label{PCH-4}
\mathrm{Pr}[D^\prime(H(U_{N}))=1]=\mathrm{Pr}[D(U_{2N})=1].
\end{equation}
 From (\ref{PCH-2}) and (\ref{PCH-4}), one can deduce that
there exists a polynomial time probabilistic
algorithm $D^\prime$, a positive polynomial $p$, such that for all $k_0$ there exists
$N\geq \frac{k_0}{2}$ satisfying 
$$| \mathrm{Pr}[D(H(U_{N}))=1]-\mathrm{Pr}[D(U_{kN}=1]|\geq \frac{1}{p(2N)},$$
proving that $H$ is not secure, which is a contradiction. 
\end{proof}

We should wonder whether such a result continues to hold for other versions of
CIPRNGs presented in this manuscript. We once thought until Philippe Guillot 
provides evidences that, when considering 2 generators as input, if only one of
the two generators is cryptographically secure, then the CIPRNG cannot be secure.


