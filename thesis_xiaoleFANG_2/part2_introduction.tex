\part{Pseudorandom Number Generator Based on Chaotic Iteration}
\label{Pseudo Random Number Generator Based on Chaotic Iteration}
\chapter{Computer Science RNGs: an Introduction}
\label{Introduction}
\minitoc
Alone with the rapid development of Internet and universal application of multimedia technology, multimedia data including audio, image, and video has been transmitted over insecure channels, it implies the need to protect data and privacy in digital world. This development has revealed new major security issues. For example, new security concerns have recently appeared because of the evolution of the Internet to support such activities as e-Voting, VoD, and digital rights management~\cite{Zhu200675}. The random number generators  are very important cryptographic primitive widely used in the Internet security, because they are fundamental in cryptosystems and information hiding schemes. 

RNGs are widely used in science and technology, it is a critical component in modern cryptographic systems, communication systems, statistical simulation systems, and any scientific area incorporating Monte Carlo methods and many others \cite{quantum,communication,cryptography}. The random statistical quality of the generated bit sequence is measured by two aspects: the unpredictability of the bit stream and the speed at which the random bits can be produced.  Other factors like system complexity, cost, reliability and so on, are also important for establishing successful RNGs. As stated before, there are usually two methods for random numbers generation. The first one, on which this part focuses, relates to deterministic algorithms implemented in hardware and software. In that context, pseudorandom numbers are degenerated from a single value called a ``seed'', and using an algorithm called pseudorandom number generator (PRNG~\cite{LEcuyer08}). The second approach, debated in the next part, counts on high entropy signals, either from purely non deterministic and stochastic physical phenomena, or from deterministic but chaotic dynamical systems (necessarily mixed with an unavoidable noisy and smaller compound~\cite{fast,dice}). A potential advantage of the latter physical high entropy signal, resides in its deterministic features which might be used to achieve chaos synchronization as it has been already demonstrated \cite{pecora:prl90} and widely used for secure chaos communications \cite{argyris:nat05}. However, synchronization possibility of the random binary sequence extracted from the chaotic physical signal is still an open problem, which resolution could lead to the efficient and practical use of the Vernam cypher.

For the PRNGs algorithms, they are majorly defined by a deterministic recurrent sequence in a finite state space, usually a finite field or ring, and an output function mapping each state to an input value. This is often either a real number in the interval $(0,1)$ or an integer in some finite range~\cite{LEcuyer08}. Such PRNGs can be easily implemented in any computational platform, however they suffer from the vulnerability that the future sequence can be deterministically computed if the seed or internal state of the algorithm is discovered. Additionally to their reproducible character, the main advantages of PRNGs is that no hardware cost is added and the speed is only counted on processing hardware. In 
a cryptographic context, these algorithms are developed to prevent guessing of the initial conditions, thus their speed are slowed down due to the increased complexity of such precaution.

Recently, some researchers have demonstrated the possibility to use chaotic dynamical systems as RNGs to reinforce the security of cryptographic algorithm, due to the unpredictability and distort-like property of chaotic dynamical systems~\cite{Falcioni2005,Cecen2009,PO2004}. These attempts are related to the hypothesis that digital chaotic systems can possibly reinforce the security of cryptographic algorithms, because the behaviors of such systems are very similar to those of physical noise sources~\cite{Schuster1984}. For instance, in \cite{cite-key}, chaos has been applied to strengthen some optical communications. More generally, the random-like and unpredictable dynamics of chaotic systems, their inherent determinism and simplicity of realization suggest their potential for exploitation as RNGs.

In chaotic cryptography, there are two main design paradigms: in the first paradigm chaotic cryptosystems are realized in analog circuits (mainly based on chaos synchronization~\cite{PhysRevLett.64.821}) whereas in the second paradigm chaotic cryptosystems are realized in digital circuits or computers (synchronization is not an issue). Generally speaking, synchronization based chaotic cryptosystems are generally designed for securing communications though noisy channels, and they cannot be directly extended to design digital ciphers in pure cryptography. However, some cryptanalytic works have shown that most synchronization based chaotic cryptosystems are not really secure, since it is possible to extract some information on the secret chaotic parameters~\cite{BethLaMa94}. Therefore, although chaos synchronization is still actively studied in research of secure communications, as it is related in the next part
of this manuscript, the ideas underlying these approaches still remain disputed by conventional cryptographers. Since this dissertation is devoted to researches lying between chaotic cryptography and traditional cryptography, we will thus focus first on the second paradigm in this part of the dissertation. 

Even though chaotic systems exhibit random-like behavior, they are not necessarily cryptographically secure in their discretized form, see e.g.~\cite{HabutsuNSM91,Biham91cryptanalysisof}. The reason partly being that discretized chaotic functions do not automatically yield sufficiently complex behavior of the corresponding binary functions, which is, roughly speaking, a prerequisite for cryptographic security. It is therefore essential that the complexity of the binary functions is considered in the design phase such that necessary modifications can be made. Moreover, many suggested PRNG based on chaos suffer from reproducibility problems of the keystream due to the different handling of floating-point numbers on various processors, see e.g. \cite{Matthews:1984}.
Indeed, as outlined above, chaotic dynamical systems are usually continuous and hence defined on the real numbers domain. The transformation from real numbers to integers may lead to the loss of the chaotic behavior. The conversion to integers needs a rigorous theoretical foundation.

In this part, some new chaotic pseudorandom bit generator is presented, which can also be used to obtain numbers uniformly distributed between 0 and 1\footnote{Indeed, these bits can be grouped $n$ by $n$, to obtain the floating part of $x \in [0,1]$ represented in binary numeral system}. These generators are based on discrete chaotic iterations which satisfy Devaney's definition of chaos~\cite{guyeux09}. A rigorous  framework is introduced, where topological chaotic properties of the generator are shown. 
The design goal of  these generators was to take advantage of the random-like properties of real-valued chaotic maps and, at the same time, secure optimal cryptographic properties. More precisely, the design was initiated by constructing a chaotic system on the integers domain instead of the real numbers domain.

The quality of a PRNG is proven both by theoretical foundations and empirical validations. Various statistical tests are available in the literature to check empirically the statistical quality of a given sequence. The most famous and important batteries of tests for evaluating PRNGs have been presented previously. They are respectively the TestU01~\cite{Lecuyer2009}, NIST (National Institute of Standards and Technology of the U.S. Government), DieHARD suites~\cite{ANDREW2008,Marsaglia1996}, and Comparative test parameters~\cite{Menezes1997}. For various reasons, a generator can behave randomly according to some of these tests, but it can fail to pass some other tests. So to pass a number of tests as large as possible is important to improve the confidence put in the randomness of a given generator~\cite{Turan2008}. We will now introduce the PRNGs family on which my computer science researches have been focused.

