\chapter{Conclusions and Future Work}
\label{Conclusions and Future Work}
\minitoc

\section{Conclusions}
Chaos, being a special class of nonlinear dynamics, has aroused a lot of interests since its
discovery. It is well-known with its distinct characteristics, such as the exhibition of
random-like behaviors, the sensitive dependence on initial conditions and control parameters,
the ergodic and mixing nature, and so on.
Recently, many research works have been witnessed where the chaotic dynamics are
applied for cryptographical applications and designs. Being a major element in cryptography,
the nature of randomness and chaos have been related, and the blossom of chaos-based
random number generators and cryptographical systems is observed, even though it causes
considerable controversy to comment that chaos is a very good candidate.
In most of the designs, the generation of chaos is obtained by a recurrence relationship
or a set of state differential equations, in which perfect model is assumed. It always requires a
continuous space domain so that the actual chaotic dynamics can be observed. However, this
compact space domain implies the use of real-number or infinite bit representation in system
realization.
In this paper, a family of pseudorandom generators called CI PRNGs is detailed and evaluated. 
The generation of pseudorandom numbers is realized by combining two well-known PRNGs with chaotic iterations.
By doing so, we obtain fast generators which additionally satisfy chaotic properties.
In addition to passing the NIST and DieHARD tests suites, some generators in this family successfully pass all the stringent TestU01 battery of tests.
The randomness and disorder generated by these algorithms have been evaluated, leading to the conclusion that these generators possess properties such that they can be considered as candidates for a large variety of applications in computer science security field.
An application example in this field is finally given at the end of this paper.



\section{Future Work}
In future work, we will continue to improve the speed and security of this family of PRNGs, by exploring new strategies and iteration functions. 
Their chaotic behavior will be studied more deeply by using various tools provided by the mathematical theory of chaos. 
New statistical tests will be used to compare these PRNGs to existing ones.
Additionally a probabilistic study of their security will be done.
Lastly, new applications in computer science will be proposed, as in the Internet security field.