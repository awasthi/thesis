\chapter{Appendix}
In a cryptographic context, a pseudorandom generator is a deterministic algorithm $G$ transforming strings into strings and such that, for any seed 
$s$ of length m, $G(s)$ (the output of $G$ on the input $s$) has size $l_G(m)$ with $l_G(m) > m$. The notion of secure 
PRNGs can now be defined as follows.

\begin{definition}
\label{CSPRNG}
A cryptographic PRNG $G$ is secure if for any probabilistic polynomial time algorithm D, for any positive polynomial p, 
and for all sufficiently large m's,  
\begin{equation}
\left|Pr[D(G(U_m))=1]-Pr[D(U_{l_G(m)})=1]\right|<\frac{1}{p(m)},
\end{equation}
where $U_r$ is the uniform distribution over $\{0, 1\}^r$ and the probabilities are taken over $U_m$, 
$U_{l_G(m)}$ as well as over the internal coin tosses of $D$.
\end{definition}

Intuitively, it means that there is no polynomial time algorithm that can distinguish a perfect uniform 
random generator from $G$ with a non negligible probability. Note that it is quite easily possible to change 
the function $l$ into any polynomial function $l'$ satisfying $l'(m)>m$. 